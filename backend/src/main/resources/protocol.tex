\documentclass[a4paper,12pt]{article} %our class is article

\usepackage{cmap}			% поиск в PDF
\usepackage[T2A]{fontenc}	% кодировка
\usepackage[utf8]{inputenc} % кодировка исходного текста
\usepackage[russian]{babel}	%локализация и переносы
\usepackage{ragged2e}		%для выравнивания по ширине \justifying
\usepackage[left=25mm, top=5mm, right=10mm, bottom=20mm, nohead, nofoot]{geometry}
\usepackage{setspace}

\begin{document}
	
\thispagestyle{empty}
	\begin{center} \small
		\textbf
		{МИНИСТЕРСТВО ОБРАЗОВАНИЯ И НАУКИ РОССИЙСКОЙ ФЕДЕРАЦИИ}\\
		\vspace{0.1cm}
		\tiny 
		ФЕДЕРАЛЬНОЕ ГОСУДАРСТВЕННОЕ АВТОНОМНОЕ ОБРАЗОВАТЕЛЬНОЕ УЧРЕЖДЕНИЕ
	    ВЫСШЕГО  ОБРАЗОВАНИЯ\\
		\small 
		«Национальный исследовательский ядерный университет «МИФИ»\\
		\textbf
		{Обнинский институт атомной энергетики} – \\
	    \footnotesize	
		филиал федерального государственного автономного образовательного учреждения высшего образования \\«Национальный исследовательский ядерный университет «МИФИ»\\
		\textbf{(ИАТЭ НИЯУ МИФИ)}
	\end{center}
	
\begin{center}
		ПРОТОКОЛ\\
		от \underline{@date } №\ \underline{@number }\\
		заседания Государственной экзаменационной комиссии \\		
\end{center}

\begin{center} 
	по направлению подготовки:
	\textbf
	{ \underline{09.03.02 Информационные системы и технологии} }
\end{center}

\begin{flushleft} \justifying
	вид государственного аттестационного испытания:
	\underline{защита выпускной квалификационной работы}\\
	по профилю 
	\underline{«Информационные технологии»}
\end{flushleft}

\begin{center} 		
	\begin{tabular}{lll}
		\vspace{0.2cm}
		ПРИСУТСТВОВАЛИ:\\
		\vspace{0.4cm}
		Председатель:  \hspace{2cm} 	& \underline{@president}  \\
		Члены комиссии:\hspace{2cm} 	& \underline{@member1}  \\
					   \hspace{2cm}  	& \underline{@member2}  \\
					   \hspace{2cm}		& \underline{@member3}  \\
					   \hspace{2cm}		& \underline{@member4}  \\
		\ \\ %Чтоб перед секретарём был пробел
		\vspace{0.4cm}			
		Секретарь:     \hspace{2cm}		& \underline{@secretary}  \\	    
	\end{tabular}
\end{center}
	\vspace{-0.3cm}
	
	\hspace{1cm}РАССМАТРИВАЛИ выпускную квалификационную работу 
	\underline{@student,}\\
	
	\vspace{-0.4cm}
	\hspace{1cm}студента (ки) отделения \underline{Интеллектуальных кибернетических систем}

    \vspace{0.2cm}
	\hspace{1cm}На тему: 
	

	\hspace{1cm}\underline{@title}
	
    \vspace{0.2cm}
	\hspace{1cm}Выпускная квалификационная работа выполнена под руководством
	
	
    \hspace{1cm}\underline{@mentor}\\

\hspace{-0.25cm}В комиссию представлены следующие материалы:
\vspace{-0.25cm}
\begin{enumerate}	
	\item  	справка деканата о сданных экзаменах и зачетах;
	\vspace{-0.3cm}
	\item  	пояснительная записка к выпускной квалификационной работе на \makebox[10mm]{\hrulefill} страницах;
	\vspace{-0.3cm}
	\item   чертежи (таблицы) и\textbackslash{}или презентации работы на \makebox[10mm]{\hrulefill}		 листах и\textbackslash{}или 	\makebox[10mm]{\hrulefill}		слайдах;
	\vspace{-0.3cm}
	\item 	отзыв руководителя выпускной квалификационной работы;
	\vspace{-0.3cm}
	\item	рецензия на выпускную квалификационную работу;
	\vspace{-0.3cm}
	\item  	скриншот результатов проверки на заимствование (плагиат).
\end{enumerate}

\hspace{-0.15cm}Сообщение о выпускной квалификационной работе длилось \underline{@time} мин., после чего\\
 студенту (ке) были заданы следующие вопросы:\\
\vspace{-0.5cm} 
\begin{flushleft}
	  	1. @question1
	  	\smallskip\hrule
	  	\vspace{0.3cm} 
	  	2. @question2
	  	\smallskip\hrule
	  	\vspace{0.3cm}
	  	3. @question3 
	  	\smallskip\hrule
	  	\vspace{0.3cm} 
	  	4. @question4
	  	\smallskip\hrule
	  	\vspace{0.3cm} 
	  	5. @question5
	  	\smallskip\hrule
	  	\vspace{0.3cm} 
	  	6. @question6
	  	\smallskip\hrule
	  	\vspace{0.3cm} 	 	
	  	
\end{flushleft}

\pagebreak

\hfill \break
\hfill \break
%Вторая страница
\thispagestyle{empty}
 
	Общая характеристика ответов студента (ки) на заданные ему вопросы и замечания\\ рецензента  
	\vspace{0.4cm}
	\smallskip\hrule
	\vspace{0.4cm} 
	\smallskip\hrule
	\vspace{0.4cm} 
	\smallskip\hrule
	\vspace{0.4cm} 
	\smallskip\hrule
	\vspace{0.4cm} 
	\smallskip\hrule
	\vspace{0.4cm} 
	\smallskip\hrule
	\vspace{0.4cm} 

\par\medskip
Итоги освоения студентом (кой) образовательной программы (средний балл)

Подготовка студента (ки):
\vspace{-0.2cm}
\begin{enumerate}
	\item оценки 5: \underline{@mark5} \%
	\vspace{-0.35cm}
	\item оценки 4: \underline{@mark4} \%
	\vspace{-0.35cm}
	\item оценки 3: \underline{@mark3} \%
	\vspace{-0.35cm}
	\item средний балл: \underline{@markAverage} \%
	\vspace{-0.35cm}
\end{enumerate}
\par\bigskip
Руководитель выпускной квалификационной работы 
\underline{@mentor}\\
считает, что выпускная квалификационная работа студента (ки) заслуживает \\
оценки  \makebox[10mm]{\hrulefill}\makebox[10mm]{\hrulefill}\makebox[10mm]{\hrulefill}
\par\medskip
Рецензент \underline{@reviewer} считает, что выпускная квалификационная\\ работа заслуживает оценки \makebox[10mm]{\hrulefill}\makebox[10mm]{\hrulefill}\makebox[10mm]{\hrulefill}
\par\medskip

Государственная экзаменационная комиссия РАССМОТРЕЛА:

- результаты выполнения студентом (кой) НИЯУ МИФИ \underline{@student}\\
 по направлению подготовки:\underline{09.03.02 Информационные системы и технологии}  (форма обучения\\
  очная) учебного плана и освоения образовательной программы, отраженные в учебной карточке;

- итоги защиты выпускной квалификационной работы;

- иные документы, представленные отделением Интеллектуальных кибернетических систем

\par\smallskip

ПРИНЯЛА РЕШЕНИЕ:

\begin{enumerate}
	\item 	Признать, что студент (ка) \underline{@student} защитил (а) \\
	выпускную квалификационную работу с оценкой \makebox[10mm]{\hrulefill}\makebox[10mm]{\hrulefill}, ECTS \makebox[10mm]{\hrulefill}. 
	\vspace{-0.25cm}
	\item 	Присвоить выпускнику (це) \underline{@student} квалификацию (степень)\\ «бакалавр» по направлению подготовки \underline{09.03.02 Информационные системы и технологии} 
	\vspace{-0.75cm}
	\item 	Выдать диплом государственного образца с отличием / без отличия выпускнику (це) \\
	НИЯУ МИФИ \underline{@student} по направлению подготовки\\
	\underline{09.03.02 Информационные системы и технологии} 
	\vspace{-0.20cm}
	\item 	Рекомендовать выпускника (цу) \underline{@student}  для поступления\\
	 в магистратуру
	\vspace{-0.25cm}
\end{enumerate}
\par\bigskip
\begin{center} 	
	\begin{tabular}{lll}
		\vspace{0.3cm}
		Председатель ГЭК:\hspace{1cm} &\makebox[40mm]{\hrulefill} & @president  \\
		Члены комиссии:  \hspace{1cm} &\makebox[40mm]{\hrulefill}\par\medskip & @member1 \\
		
						 \hspace{1cm} &\makebox[40mm]{\hrulefill}\par\medskip & @member2  \\
					   	 \hspace{1cm} &\makebox[40mm]{\hrulefill}\par\medskip & @member3  \\
						 \hspace{1cm} &\makebox[40mm]{\hrulefill}\par\medskip & @member4  \\
		\ \\ %Чтоб перед секретарём был пробел
		\vspace{0.3cm}			
		Секретарь ГЭК:   \hspace{1cm} & \makebox[40mm]{\hrulefill}  &@secretary  \\		
	\end{tabular}
\end{center}

\end{document}